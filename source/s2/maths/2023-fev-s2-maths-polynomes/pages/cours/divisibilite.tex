\begin{theorem}[red]{Définition 2.1 : Divisibilité de  polynômes}
    Soient $A, B \in \ensK[X]$. \\
    $A$ \textbf{divise} $B$ si et seulement si $\exists Q \in \ensK[X], B = AQ$. \\
    On note alors $A \ | \ B$.
\end{theorem}

\begin{theorem}[blue]{Propriété 2.2 : Divisibilité}
    Soient $A, B, C$ trois polynômes non nuls à coefficients dans $\ensK$.

    \begin{itemize}
        \item Si $A \ | \ B$ et $B \ | \ C$, alors $ A \ | \ C$.
        \item Si $A \ | \ B$ et $B \ | \ A$, alors $A = \lambda B$ avec $\lambda \in \ensK$.
        \item Si $A \ | \ B$ et $A \ | \ C$, alors $\forall (P, Q) \in \ensK[X]^2, A \ | \ BP + CQ$.
    \end{itemize}
\end{theorem}

\begin{theorem}[blue]{Propriété 2.3 : Divisibilité et degré}
    Soient $A, B$ deux polynômes à coefficients dans $\ensK$.

    \begin{itemize}
        \item Si $A \ | \ B$ et $B \neq 0$, alors $deg(A) \leq deg(B)$.
        \item Si $A \ | \ B$ et $deg(A) = deg(B)$, alors $A = \lambda B$ avec $\lambda \in \ensK^*$
    \end{itemize}
\end{theorem}

\begin{theorem}[orange]{Théorème 2.4 : Division euclidienne}
    Soient $A, B$ deux polynômes à coefficients dans $\ensK$ avec $B \neq 0$.

    $$
        \exists ! \ (Q, R) \in \ensK[X]^2, A = BQ + R \hspace{10px} \text{avec } deg(R) < deg(B)
    $$

    \noindent Le polynôme $Q$ est le quotient et le polynôme $R$ est le reste de la \textbf{division euclidienne} de $A$ par $B$.
\end{theorem}

\begin{theorem}[blue]{Propriété 2.5 : Divisibilité et division euclidienne}
    Soient $A, B \in \ensK[X]$. \\
    $B \ | \ A$ si et seulement si le reste de la division euclidienne de $A$ par $B$ est nul.
\end{theorem}

\begin{theorem}[black]{Exemple de division euclidienne}
    \begin{center}
        \polylongdiv[style=D]{2X^4 - X^2 + X - 5}{X^2 + X - 2}
    \end{center}

    \vspace{5px}

    \noindent Donc, $2X^4 - X^2 + X - 5 = (X^2 + X - 2)(2X^2 - 2X + 5) - 8X + 5$.
\end{theorem}