\documentclass[a4paper, 14px, fleqn]{article}

% Packages importés
\usepackage[utf8]{inputenc} % Encodage des entrées
\usepackage[T1]{fontenc} % Encodage de la police
\usepackage[french]{babel} % Support pour la langue
\usepackage{fancyhdr} % En-têtes et pieds de page
\usepackage{graphicx} % Ajouter des images
\usepackage{hyperref} % Ajouter les liens dans la table de matière
\usepackage{colortbl} % Ajouter des couleurs dans un tableau
\usepackage{amsmath,amssymb,polynom} % Ajouter les maths
\usepackage{mathtools} % Ajouter des maths
\usepackage{tcolorbox} % Ajouter les cadres
\usepackage{stmaryrd} % Ajouter les doubles brackets
\usepackage{tikz} % Ajouter les schémas
\usetikzlibrary{matrix,arrows,decorations.pathmorphing}
\usepackage{verbatim} % Ajouter les matrices
\usepackage{nicematrix} % Ajouter les matrices
\usepackage[overload]{empheq}

% Utiliser de bonnes marges
\setlength{\hoffset}{-18px}
\setlength{\oddsidemargin}{0px} % Marge gauche sur pages impaires
\setlength{\evensidemargin}{9px} % Marge gauche sur pages paires
\setlength{\marginparwidth}{54px} % Largeur de note dans la marge
\setlength{\textwidth}{481px} % Largeur de la zone de texte (17cm)
\setlength{\voffset}{-18px} % Bon pour DOS
\setlength{\marginparsep}{7px} % Séparation de la marge
\setlength{\topmargin}{0px} % Pas de marge en haut
\setlength{\headheight}{13px} % Haut de page
\setlength{\headsep}{10px} % Entre le haut de page et le texte
\setlength{\footskip}{27px} % Bas de page + séparation
\setlength{\textheight}{708px} % Hauteur de la zone de texte (25cm)

% Ajouter les ensembles
\newcommand{\ensR}{\mathbb{R}}
\newcommand{\ensC}{\mathbb{C}}
\newcommand{\ensK}{\mathbb{K}}
\newcommand{\ensN}{\mathbb{N}}

% Theoremes
\usepackage{tikz, tkz-tab}
\usepackage[framemethod=tikz]{mdframed}
\usetikzlibrary{shapes.misc}

% Les arguments des theoremes importants
\makeatletter
\def\mdf@@title{}
\define@key{mdf}{title}{\def\mdf@@title{#1}}
\def\mdf@@color{}
\define@key{mdf}{color}{\def\mdf@@color{#1}}

% Le style des theoremes
\mdfdefinestyle{mytheo}{
    linewidth=1pt,
    innertopmargin=1.5\baselineskip,
    innerbottommargin=0.7\baselineskip,
    roundcorner=5pt,
    backgroundcolor=\mdf@@color!05,
    linecolor=\mdf@@color,
    singleextra={
        \node[xshift=10pt,thick,draw=\mdf@@color,fill=\mdf@@color!65,rounded 
            corners,text=white,anchor=west] at (P-|O) %
        {\strut{\bfseries\mdf@@title}};
    }
}

% Pour definir le theme comme servant d'environement
\newmdenv[style=mytheo]{theor}

% L'environement mais cette fois il peut prendre les parametres
\newenvironment{theorem}[2][]
  {\begin{theor}[color=#1, title=#2]}
  {\end{theor}}

\makeatother


% Exemples
\newcommand{\exobox}[1]
{
    \begin{tcolorbox}
    {#1}
    \end{tcolorbox}
}

% Mettre en place les en-têtes et les pieds de pages
\pagestyle{fancy}
\fancyhf{}
\lhead{BackToBasics}
\rhead{Mathématiques S2}
\lfoot{\includegraphics[scale=0.15]{img/b2b.png}}
\cfoot{\thepage}
\rfoot{\includegraphics[scale=0.01]{img/epita.png}}

\begin{document}

    % Première page du document
    \vspace*{\stretch{1}}
\begin{center}

    \vspace{15px}
    \Huge{\textbf{Mathématiques}}\\
	\huge{\textbf{Polynômes}}\\
	\vspace{10px}
	\Large{\textbf{BackToBasics}}\\
	\textit{11 février 2023}

    \vspace{25px}
	\includegraphics{img/b2b.png}
	
\end{center}
\vspace*{\stretch{1}}
    \clearpage
    
    % Page de table de matières
    \tableofcontents
    \clearpage
    
    % Page de cours
    \section{Cours}
    \noindent Dans ce cours, nous allons revoir tout ce qui concerne les polynômes. Si vous avez des questions sur n'importe quel point du cours, n'hésitez pas à poser votre question dans le salon dédié ou par messages privés sur Discord à \textbf{Swarwerth\#2943}.

\subsection{Définitions et vocabulaire}
\begin{theorem}[red]{Définition 1.1 : Polynôme}
    Un \textbf{polynôme} à coefficients dans $\ensK$ est une expression de la forme

    $$
        P(X) = a_nX^n + a_{n - 1}X^{n - 1} + ... + a_2X^2 + a_1X + a_0
    $$

    \noindent avec $n \in \ensN$ le \textbf{degré} de $P$ et les \textbf{coefficients} du polynôme $a_0, a_1, ..., a_n \in \ensK$ (où $\ensK$ désigne $\ensR$ ou $\ensC$). \\
    On dit alors que $P \in \ensK[X]$.
\end{theorem}

\begin{theorem}[blue]{Propriété 1.2 : Opérations sur les polynômes et degré}
    Soient $P, Q \in \ensK[X]$, \\

    \noindent $deg(P \times Q) = deg(P) + deg(Q)$. \\
    \noindent $deg(P + Q) = max(deg(P), \ deg(Q))$.
\end{theorem}

\begin{theorem}[black]{Exemple d'opérations sur les polynômes}
    Soient $P(X) = 3X^3 + 4X + 5$ et $Q(X) = 4X^2 + 3$.

    \begin{align*}
        (P + Q)(X) &= 3X^3 + 4X + 5 + 4X^2 + 3 \\
        &= 3X^3 + 4X^2 + 4X + 8
    \end{align*}

    \begin{align*}
        (P \times Q)(X) &= (3X^3 + 4X + 5)(4X^2 + 3) \\
        &= 3X^3 \times 4X^2 + 3X^3 \times 3 + 4X \times 4X^2 + 4X \times 3 + 5 \times 4X^2 + 5 \times 3 \\
        &= 12X^5 + 9X^3 + 16X^3 + 12X + 20X^2 + 15 \\
        &= 12X^5 + 25X^3 + 20X^2 + 12X + 15
    \end{align*}
\end{theorem}
\clearpage

\subsection{Divisibilité et division euclidienne}
\begin{theorem}[red]{Définition 2.1 : Divisibilité de  polynômes}
    Soient $A, B \in \ensK[X]$. \\
    $A$ \textbf{divise} $B$ si et seulement si $\exists Q \in \ensK[X], B = AQ$. \\
    On note alors $A \ | \ B$.
\end{theorem}

\begin{theorem}[blue]{Propriété 2.2 : Divisibilité}
    Soient $A, B, C$ trois polynômes non nuls à coefficients dans $\ensK$.

    \begin{itemize}
        \item Si $A \ | \ B$ et $B \ | \ C$, alors $ A \ | \ C$.
        \item Si $A \ | \ B$ et $B \ | \ A$, alors $A = \lambda B$ avec $\lambda \in \ensK$.
        \item Si $A \ | \ B$ et $A \ | \ C$, alors $\forall (P, Q) \in \ensK[X]^2, A \ | \ BP + CQ$.
    \end{itemize}
\end{theorem}

\begin{theorem}[blue]{Propriété 2.3 : Divisibilité et degré}
    Soient $A, B$ deux polynômes à coefficients dans $\ensK$.

    \begin{itemize}
        \item Si $A \ | \ B$ et $B \neq 0$, alors $deg(A) \leq deg(B)$.
        \item Si $A \ | \ B$ et $deg(A) = deg(B)$, alors $A = \lambda B$ avec $\lambda \in \ensK^*$
    \end{itemize}
\end{theorem}

\begin{theorem}[orange]{Théorème 2.4 : Division euclidienne}
    Soient $A, B$ deux polynômes à coefficients dans $\ensK$ avec $B \neq 0$.

    $$
        \exists ! \ (Q, R) \in \ensK[X]^2, A = BQ + R \hspace{10px} \text{avec } deg(R) < deg(B)
    $$

    \noindent Le polynôme $Q$ est le quotient et le polynôme $R$ est le reste de la \textbf{division euclidienne} de $A$ par $B$.
\end{theorem}

\begin{theorem}[blue]{Propriété 2.5 : Divisibilité et division euclidienne}
    Soient $A, B \in \ensK[X]$. \\
    $B \ | \ A$ si et seulement si le reste de la division euclidienne de $A$ par $B$ est nul.
\end{theorem}

\begin{theorem}[black]{Exemple de division euclidienne}
    \begin{center}
        \polylongdiv[style=D]{2X^4 - X^2 + X - 5}{X^2 + X - 2}
    \end{center}

    \vspace{5px}

    \noindent Donc, $2X^4 - X^2 + X - 5 = (X^2 + X - 2)(2X^2 - 2X + 5) - 8X + 5$.
\end{theorem}
\clearpage

\subsection{Racines d'un polynôme}
\begin{theorem}[red]{Définition 3.1 : Racine d'un polynôme}
    Soit $P \in \ensK[X]$. \\
    On appelle \textbf{racine du polynôme} $P$ tout élément $\alpha \in \ensK$ tel que $P(\alpha) = 0$.
\end{theorem}

\begin{theorem}[blue]{Propriété 3.2 : Racine et divisibilité}
    Soit $P \in \ensK[X]$. \\
    $\alpha \in \ensK$ est une racine de $P$ si et seulement si $(X - \alpha) \ | \ P$. 
\end{theorem}

\begin{theorem}[red]{Définition 3.3 : Multiplicité d'une racine}
    Soient $P \in \ensK[X]$, $\alpha \in \ensK$ et $n \in \ensN*$. \\
    On dit que $\alpha$ est une \textbf{racine de multiplicité} $n$ (ou d'ordre $n$) si $(X - \alpha)^n \ | \ P$ et que $(X - \alpha)^{n + 1} \ \not| \ P$. \\

    \noindent On parle de \textbf{racine simple} lorsque $n = 1$ et de \textbf{racine double} lorsque $n = 2$.
\end{theorem}

\begin{theorem}[blue]{Propriété 3.4 : Propriétés d'une racine}
    Soient $P \in \ensK[X]$ et $\alpha \in \ensK$.
    Il existe une équivalence entre ces trois points :

    \begin{itemize}
        \item $\alpha$ est une racine de multiplicité $n$ de $P$
        \item $\exists Q \in \ensK[X], P = (X - \alpha)^nQ$  avec $Q(\alpha) \neq 0$
        \item $P(\alpha) = P'(\alpha) = ... = P^{(n - 1)}(\alpha) = 0$ et $P^{(n)} \neq 0$
    \end{itemize}
\end{theorem}

\begin{theorem}[orange]{Théorème 3.5 : Théorème de d'Alembert-Gauss}
    Tout polynôme à coefficients complexes de degré $n \geq 1$ a au moins une racine dans $\ensC$. Il admet exactement $n$ racines si on compte chaque racine avec sa multiplicité.
\end{theorem}

\begin{theorem}[black]{Exemple d'un polynôme du second degré}
    Soit $P(X) = aX^2 + bX + c$ un polynôme de degré 2 à coefficients réels, avec $a \neq 0$.

    \begin{itemize}
        \item Si $\Delta = b^2 - 4ac > 0$, alors $P$ admet 2 racines réelles distinctes $\displaystyle \frac{-b + \sqrt{\Delta}}{2a}$ et $\displaystyle \frac{-b - \sqrt{\Delta}}{2a}$.
        \item Si $\Delta < 0$, alors $P$ admet 2 racines complexes distinctes $\displaystyle \frac{-b + i\sqrt{| \Delta |}}{2a}$ et $\displaystyle \frac{-b - i\sqrt{| \Delta |}}{2a}$.
        \item Si $\Delta = 0$, alors $P$ admet une racine double $\displaystyle \frac{-b}{2a}$.
    \end{itemize}
\end{theorem}

\begin{theorem}[orange]{Théorème 3.6 : Degré et racines}
    Soit $P \in \ensK[X]$ de degré $n \geq 1$. \\
    Alors, $P$ admet au plus $n$ racines dans $\ensK$.
\end{theorem}
\clearpage

\subsection{Factorisation}
\begin{enumerate}
    \item Factoriser $X^4 - 1$ dans $\ensC[X]$.
    \item Factoriser en utilisant une identification $X^3 - 2X^2 - 5X + 6$ sachant que $1$ est une racine de ce polynôme.
    \item Calculer la division euclidienne de $X^3 - X^2 - 10X - 8$ par $X + 1$ et en déduire une factorisation.
\end{enumerate}
\clearpage
    \clearpage
    
    % Page d'exercices
    \section{Exercices}
    \noindent Nous avons essayé de mettre l'ensemble des connaissance à savoir pour les rattrapages au niveau du chapitre sur les polynômes. Évidemment, vous avez le droit de relire le cours ou de poser des questions. Ces exercices permettront de pointer les lacunes que vous pourriez avoir.

\subsection{Exercice 1 : Division euclidienne}
\begin{enumerate}
    \item Soit $(A, B) \in \ensR[X]^2$ avec $B \neq 0$. Énoncer le théorème de la division euclidienne de $A$ par $B$.
    \item Effectuer la division euclidienne de $X^4 + 5X^3 + 12X^2 + 19X - 7$ par $X^2 + 3X - 1$.
    \item Effectuer la division euclidienne de $X^4 - 4X^3 - 9X^2 + 27X + 38$ par $X^2 - X - 7$
\end{enumerate}
\vspace{10px}

\subsection{Exercice 2 : Divisibilité}
\noindent Soit $n \geq 2$,

\begin{enumerate}
    \item Montrer que le polynôme $P(X) = (X + 1)^{2n} - 1$ est divisible par $X^2 + 2X$.

    \begin{theorem}[black]{Divisibilité de $P(X)$ par $X^2 + 2X$}
        On cherche

        \begin{align*}
            X^2 + 2X \ | \ P(X) &\Longleftrightarrow X(X + 2) \ | \ P(X) \\
            &\Longleftrightarrow (X - 0) \ | \ P \text{ et } (X + 2) \ | \ P(X) \\
            &\Longleftrightarrow P(0) = P(-2) = 0
        \end{align*}

        \begin{align*}
            P(0) &= (0 + 1)^{2n} - 1 = 1 - 1 = 0 \\
            P(-2) &= (-2 + 1)^{2n} - 1 = 1 - 1 = 0
        \end{align*}

        Donc, $X^2 + 2X \ | \ P$.
    \end{theorem}
    
    \item Montrer que $1$ est racine du polynôme $Q(X) = nX^{n + 2} - (n + 2)X^{n + 1} + (n + 2)X - n$. Quelle est sa multiplicité ?

    \begin{theorem}[black]{$1$ racine du polynôme $Q(X)$}
        $Q(1) = n \times 1^{n + 2} - (n + 2) \times 1^{n + 1} + (n + 2) \times 1 - n = n - (n + 2) + (n + 2) - n = 0$. \\
        Donc, $1$ est racine de $Q(X)$. \\

        \noindent De plus, on a :

        $$
            Q'(X) = n(n + 2)X^{n + 1} - (n + 1)(n + 2)X^n + (n + 2)
        $$

        \noindent et $Q'(1) = 0$.

        $$
            Q''(X) = n(n + 1)(n + 2)X^n - n(n + 1)(n + 2)X^{n - 1}
        $$

        \noindent et $Q''(1) = 0$.

        $$
            Q^{(3)}(X) = n^2(n + 1)(n + 2)X^{n - 1} - (n - 1)n(n + 1)(n + 2)X^{n - 2}
        $$

        \noindent et $Q^{(3)}(1) = n(n + 1)(n + 2)$. \\
        Or, $n \geq 2$, \\
        Donc, $P^{(3)} \neq 0$. \\

        \noindent Ainsi, $1$ est racine de multiplicité $3$ de $P$.
        
    \end{theorem}
    
\end{enumerate}
\vspace{10px}

\subsection{Exercice 3 : Racines et polynômes irréductibles}
\noindent On considère le polynôme $P(X) = X^4 + 2X^3 - X - 2$

\begin{enumerate}
    \item Montrer que $1$ et $-2$ sont racines de $P$.

    \begin{theorem}[black]{Racines de $P$}
        \begin{align*}
            P(1) &= 1^4 + 2 \times 1^3 - 1 - 2 = 0 \\
            P(-2) &= (-2)^4 + 2 \times (-2)^3 - (-2) - 2 = 0
        \end{align*}

        \noindent Donc, $1$ et $-2$ sont racines de $P$.
    \end{theorem}
    
    \item En vous aidant d'une division euclidienne, factoriser $P$ en produit de polynômes irréductibles dans $\ensR[X]$.

    \begin{theorem}[black]{Factorisation de $P$}
        D'après la question (1), on a :

        \begin{align*}
            P(X) &= X^4 + 2X^3 - X - 2 \\
            &= (X - 1)(X + 2)Q(X) \hspace{10px} \text{avec } deg(Q) = 2 \\
            &= (X^2 + 2X - X - 2)Q(X) \\
            &= (X^2 + X - 2)Q(X)
        \end{align*}

        \noindent Pour déterminer $Q(X)$, nous pouvons utiliser la division euclidienne.

        \begin{center}
            \polylongdiv[style=D]{X^4 + 2X^3 - X - 2}{X^2 + X - 2}
        \end{center}

        \noindent Donc, on a :

        \begin{align*}
            P(X) &= (X^2 + X - 2)(X^2 + X + 1) \\
            &= (X - 1)(X + 2)(X^2 + X + 1)
        \end{align*}

        Sachant que $\Delta(X^2 + X + 1) < 0$, $X^2 + X + 1$ est irréductible dans $\ensR[X]$. \\
        D'où la factorisation de $P$ en produit de polynômes irréductibles dans $\ensR[X]$.
    \end{theorem}
\end{enumerate}
\vspace{10px}

\subsection{Exercice 4 : Factorisation}
\begin{enumerate}
    \item Factoriser $X^4 - 1$ dans $\ensC[X]$.
    \item Factoriser en utilisant une identification $X^3 - 2X^2 - 5X + 6$ sachant que $1$ est une racine de ce polynôme.
    \item Calculer la division euclidienne de $X^3 - X^2 - 10X - 8$ par $X + 1$ et en déduire une factorisation.
\end{enumerate}
\vspace{10px}

    \clearpage
    
    % Page de correction
    \section{Correction}
    \noindent Voici la correction des exercices du PDF. Nous vous invitons à faire les exercices avant de lire la correction. Pour toute question, merci de vous rediriger vers le cours et les professeurs présents.

\subsection{Exercice 1 : Division euclidienne}
\begin{enumerate}
    \item Soit $(A, B) \in \ensR[X]^2$ avec $B \neq 0$. Énoncer le théorème de la division euclidienne de $A$ par $B$.
    \item Effectuer la division euclidienne de $X^4 + 5X^3 + 12X^2 + 19X - 7$ par $X^2 + 3X - 1$.
    \item Effectuer la division euclidienne de $X^4 - 4X^3 - 9X^2 + 27X + 38$ par $X^2 - X - 7$
\end{enumerate}
\clearpage

\subsection{Exercice 2 : Divisibilité}
\noindent Soit $n \geq 2$,

\begin{enumerate}
    \item Montrer que le polynôme $P(X) = (X + 1)^{2n} - 1$ est divisible par $X^2 + 2X$.

    \begin{theorem}[black]{Divisibilité de $P(X)$ par $X^2 + 2X$}
        On cherche

        \begin{align*}
            X^2 + 2X \ | \ P(X) &\Longleftrightarrow X(X + 2) \ | \ P(X) \\
            &\Longleftrightarrow (X - 0) \ | \ P \text{ et } (X + 2) \ | \ P(X) \\
            &\Longleftrightarrow P(0) = P(-2) = 0
        \end{align*}

        \begin{align*}
            P(0) &= (0 + 1)^{2n} - 1 = 1 - 1 = 0 \\
            P(-2) &= (-2 + 1)^{2n} - 1 = 1 - 1 = 0
        \end{align*}

        Donc, $X^2 + 2X \ | \ P$.
    \end{theorem}
    
    \item Montrer que $1$ est racine du polynôme $Q(X) = nX^{n + 2} - (n + 2)X^{n + 1} + (n + 2)X - n$. Quelle est sa multiplicité ?

    \begin{theorem}[black]{$1$ racine du polynôme $Q(X)$}
        $Q(1) = n \times 1^{n + 2} - (n + 2) \times 1^{n + 1} + (n + 2) \times 1 - n = n - (n + 2) + (n + 2) - n = 0$. \\
        Donc, $1$ est racine de $Q(X)$. \\

        \noindent De plus, on a :

        $$
            Q'(X) = n(n + 2)X^{n + 1} - (n + 1)(n + 2)X^n + (n + 2)
        $$

        \noindent et $Q'(1) = 0$.

        $$
            Q''(X) = n(n + 1)(n + 2)X^n - n(n + 1)(n + 2)X^{n - 1}
        $$

        \noindent et $Q''(1) = 0$.

        $$
            Q^{(3)}(X) = n^2(n + 1)(n + 2)X^{n - 1} - (n - 1)n(n + 1)(n + 2)X^{n - 2}
        $$

        \noindent et $Q^{(3)}(1) = n(n + 1)(n + 2)$. \\
        Or, $n \geq 2$, \\
        Donc, $P^{(3)} \neq 0$. \\

        \noindent Ainsi, $1$ est racine de multiplicité $3$ de $P$.
        
    \end{theorem}
    
\end{enumerate}
\clearpage

\subsection{Exercice 3 : Racines et polynômes irréductibles}
\noindent On considère le polynôme $P(X) = X^4 + 2X^3 - X - 2$

\begin{enumerate}
    \item Montrer que $1$ et $-2$ sont racines de $P$.

    \begin{theorem}[black]{Racines de $P$}
        \begin{align*}
            P(1) &= 1^4 + 2 \times 1^3 - 1 - 2 = 0 \\
            P(-2) &= (-2)^4 + 2 \times (-2)^3 - (-2) - 2 = 0
        \end{align*}

        \noindent Donc, $1$ et $-2$ sont racines de $P$.
    \end{theorem}
    
    \item En vous aidant d'une division euclidienne, factoriser $P$ en produit de polynômes irréductibles dans $\ensR[X]$.

    \begin{theorem}[black]{Factorisation de $P$}
        D'après la question (1), on a :

        \begin{align*}
            P(X) &= X^4 + 2X^3 - X - 2 \\
            &= (X - 1)(X + 2)Q(X) \hspace{10px} \text{avec } deg(Q) = 2 \\
            &= (X^2 + 2X - X - 2)Q(X) \\
            &= (X^2 + X - 2)Q(X)
        \end{align*}

        \noindent Pour déterminer $Q(X)$, nous pouvons utiliser la division euclidienne.

        \begin{center}
            \polylongdiv[style=D]{X^4 + 2X^3 - X - 2}{X^2 + X - 2}
        \end{center}

        \noindent Donc, on a :

        \begin{align*}
            P(X) &= (X^2 + X - 2)(X^2 + X + 1) \\
            &= (X - 1)(X + 2)(X^2 + X + 1)
        \end{align*}

        Sachant que $\Delta(X^2 + X + 1) < 0$, $X^2 + X + 1$ est irréductible dans $\ensR[X]$. \\
        D'où la factorisation de $P$ en produit de polynômes irréductibles dans $\ensR[X]$.
    \end{theorem}
\end{enumerate}
\clearpage

\subsection{Exercice 4 : Factorisation}
\begin{enumerate}
    \item Factoriser $X^4 - 1$ dans $\ensC[X]$.
    \item Factoriser en utilisant une identification $X^3 - 2X^2 - 5X + 6$ sachant que $1$ est une racine de ce polynôme.
    \item Calculer la division euclidienne de $X^3 - X^2 - 10X - 8$ par $X + 1$ et en déduire une factorisation.
\end{enumerate}
\clearpage
    \clearpage


\end{document}
