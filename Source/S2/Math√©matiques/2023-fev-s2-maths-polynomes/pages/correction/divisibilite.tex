\noindent Soit $n \geq 2$,

\begin{enumerate}
    \item Montrer que le polynôme $P(X) = (X + 1)^{2n} - 1$ est divisible par $X^2 + 2X$.

    \begin{theorem}[black]{Divisibilité de $P(X)$ par $X^2 + 2X$}
        On cherche

        \begin{align*}
            X^2 + 2X \ | \ P(X) &\Longleftrightarrow X(X + 2) \ | \ P(X) \\
            &\Longleftrightarrow (X - 0) \ | \ P \text{ et } (X + 2) \ | \ P(X) \\
            &\Longleftrightarrow P(0) = P(-2) = 0
        \end{align*}

        \begin{align*}
            P(0) &= (0 + 1)^{2n} - 1 = 1 - 1 = 0 \\
            P(-2) &= (-2 + 1)^{2n} - 1 = 1 - 1 = 0
        \end{align*}

        Donc, $X^2 + 2X \ | \ P$.
    \end{theorem}
    
    \item Montrer que $1$ est racine du polynôme $Q(X) = nX^{n + 2} - (n + 2)X^{n + 1} + (n + 2)X - n$. Quelle est sa multiplicité ?

    \begin{theorem}[black]{$1$ racine du polynôme $Q(X)$}
        $Q(1) = n \times 1^{n + 2} - (n + 2) \times 1^{n + 1} + (n + 2) \times 1 - n = n - (n + 2) + (n + 2) - n = 0$. \\
        Donc, $1$ est racine de $Q(X)$. \\

        \noindent De plus, on a :

        $$
            Q'(X) = n(n + 2)X^{n + 1} - (n + 1)(n + 2)X^n + (n + 2)
        $$

        \noindent et $Q'(1) = 0$.

        $$
            Q''(X) = n(n + 1)(n + 2)X^n - n(n + 1)(n + 2)X^{n - 1}
        $$

        \noindent et $Q''(1) = 0$.

        $$
            Q^{(3)}(X) = n^2(n + 1)(n + 2)X^{n - 1} - (n - 1)n(n + 1)(n + 2)X^{n - 2}
        $$

        \noindent et $Q^{(3)}(1) = n(n + 1)(n + 2)$. \\
        Or, $n \geq 2$, \\

        \noindent Ainsi, $1$ est racine de multiplicité exactement $3$ de $Q$.
        
    \end{theorem}
    
\end{enumerate}