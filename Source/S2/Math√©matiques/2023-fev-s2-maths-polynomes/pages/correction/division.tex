\begin{enumerate}
    \item Soit $(A, B) \in \ensR[X]^2$ avec $B \neq 0$. Énoncer le théorème de la division euclidienne de $A$ par $B$.

    \begin{theorem}[black]{Théorème de la division euclidienne}
        Soient $A, B$ deux polynômes dans $\ensK[X]$ avec $B \neq 0$.

        $$
            \exists ! (Q, R) \in \ensK[X]^2, A = BQ + R \hspace{10px} \text{avec } deg(R) < deg(B)
        $$
    \end{theorem}
    
    \item Effectuer la division euclidienne de $X^4 + 5X^3 + 12X^2 + 19X - 7$ par $X^2 + 3X - 1$.

    \begin{theorem}[black]{Division euclidienne de $X^4 + 5X^3 + 12X^2 + 19X - 7$ par $X^2 + 3X - 1$}
        \begin{center}
            \polylongdiv[style=D]{X^4 + 5X^3 + 12X^2 + 19X - 7}{X^2 + 3X - 1}
        \end{center}

        \vspace{10px}

        \noindent Donc, on a $X^4 + 5X^3 + 12X^2 + 19X - 7 = (X^2 + 3X - 1)(X^2 + 2X + 7)$.
    \end{theorem}
    
    \item Effectuer la division euclidienne de $X^4 - 4X^3 - 9X^2 + 27X + 38$ par $X^2 - X - 7$.

    \begin{theorem}[black]{Division euclidienne de $X^4 - 4X^3 - 9X^2 + 27X + 38$ par $X^2 - X - 7$}
        \begin{center}
            \polylongdiv[style=D]{X^4 - 4X^3 - 9X^2 + 27X + 38}{X^2 - X - 7}
        \end{center}

        \vspace{10px}

        \noindent Donc, on a $X^4 - 4X^3 - 9X^2 + 27X + 38 = (X^2 - X - 7)(X^2 - 3X - 5) + X + 3$.
    \end{theorem}
\end{enumerate}